% Options for packages loaded elsewhere
\PassOptionsToPackage{unicode}{hyperref}
\PassOptionsToPackage{hyphens}{url}
%
\documentclass[
]{article}
\usepackage{amsmath,amssymb}
\usepackage{iftex}
\ifPDFTeX
  \usepackage[T1]{fontenc}
  \usepackage[utf8]{inputenc}
  \usepackage{textcomp} % provide euro and other symbols
\else % if luatex or xetex
  \usepackage{unicode-math} % this also loads fontspec
  \defaultfontfeatures{Scale=MatchLowercase}
  \defaultfontfeatures[\rmfamily]{Ligatures=TeX,Scale=1}
\fi
\usepackage{lmodern}
\ifPDFTeX\else
  % xetex/luatex font selection
\fi
% Use upquote if available, for straight quotes in verbatim environments
\IfFileExists{upquote.sty}{\usepackage{upquote}}{}
\IfFileExists{microtype.sty}{% use microtype if available
  \usepackage[]{microtype}
  \UseMicrotypeSet[protrusion]{basicmath} % disable protrusion for tt fonts
}{}
\makeatletter
\@ifundefined{KOMAClassName}{% if non-KOMA class
  \IfFileExists{parskip.sty}{%
    \usepackage{parskip}
  }{% else
    \setlength{\parindent}{0pt}
    \setlength{\parskip}{6pt plus 2pt minus 1pt}}
}{% if KOMA class
  \KOMAoptions{parskip=half}}
\makeatother
\usepackage{xcolor}
\usepackage[margin=2.54cm]{geometry}
\usepackage{color}
\usepackage{fancyvrb}
\newcommand{\VerbBar}{|}
\newcommand{\VERB}{\Verb[commandchars=\\\{\}]}
\DefineVerbatimEnvironment{Highlighting}{Verbatim}{commandchars=\\\{\}}
% Add ',fontsize=\small' for more characters per line
\usepackage{framed}
\definecolor{shadecolor}{RGB}{248,248,248}
\newenvironment{Shaded}{\begin{snugshade}}{\end{snugshade}}
\newcommand{\AlertTok}[1]{\textcolor[rgb]{0.94,0.16,0.16}{#1}}
\newcommand{\AnnotationTok}[1]{\textcolor[rgb]{0.56,0.35,0.01}{\textbf{\textit{#1}}}}
\newcommand{\AttributeTok}[1]{\textcolor[rgb]{0.13,0.29,0.53}{#1}}
\newcommand{\BaseNTok}[1]{\textcolor[rgb]{0.00,0.00,0.81}{#1}}
\newcommand{\BuiltInTok}[1]{#1}
\newcommand{\CharTok}[1]{\textcolor[rgb]{0.31,0.60,0.02}{#1}}
\newcommand{\CommentTok}[1]{\textcolor[rgb]{0.56,0.35,0.01}{\textit{#1}}}
\newcommand{\CommentVarTok}[1]{\textcolor[rgb]{0.56,0.35,0.01}{\textbf{\textit{#1}}}}
\newcommand{\ConstantTok}[1]{\textcolor[rgb]{0.56,0.35,0.01}{#1}}
\newcommand{\ControlFlowTok}[1]{\textcolor[rgb]{0.13,0.29,0.53}{\textbf{#1}}}
\newcommand{\DataTypeTok}[1]{\textcolor[rgb]{0.13,0.29,0.53}{#1}}
\newcommand{\DecValTok}[1]{\textcolor[rgb]{0.00,0.00,0.81}{#1}}
\newcommand{\DocumentationTok}[1]{\textcolor[rgb]{0.56,0.35,0.01}{\textbf{\textit{#1}}}}
\newcommand{\ErrorTok}[1]{\textcolor[rgb]{0.64,0.00,0.00}{\textbf{#1}}}
\newcommand{\ExtensionTok}[1]{#1}
\newcommand{\FloatTok}[1]{\textcolor[rgb]{0.00,0.00,0.81}{#1}}
\newcommand{\FunctionTok}[1]{\textcolor[rgb]{0.13,0.29,0.53}{\textbf{#1}}}
\newcommand{\ImportTok}[1]{#1}
\newcommand{\InformationTok}[1]{\textcolor[rgb]{0.56,0.35,0.01}{\textbf{\textit{#1}}}}
\newcommand{\KeywordTok}[1]{\textcolor[rgb]{0.13,0.29,0.53}{\textbf{#1}}}
\newcommand{\NormalTok}[1]{#1}
\newcommand{\OperatorTok}[1]{\textcolor[rgb]{0.81,0.36,0.00}{\textbf{#1}}}
\newcommand{\OtherTok}[1]{\textcolor[rgb]{0.56,0.35,0.01}{#1}}
\newcommand{\PreprocessorTok}[1]{\textcolor[rgb]{0.56,0.35,0.01}{\textit{#1}}}
\newcommand{\RegionMarkerTok}[1]{#1}
\newcommand{\SpecialCharTok}[1]{\textcolor[rgb]{0.81,0.36,0.00}{\textbf{#1}}}
\newcommand{\SpecialStringTok}[1]{\textcolor[rgb]{0.31,0.60,0.02}{#1}}
\newcommand{\StringTok}[1]{\textcolor[rgb]{0.31,0.60,0.02}{#1}}
\newcommand{\VariableTok}[1]{\textcolor[rgb]{0.00,0.00,0.00}{#1}}
\newcommand{\VerbatimStringTok}[1]{\textcolor[rgb]{0.31,0.60,0.02}{#1}}
\newcommand{\WarningTok}[1]{\textcolor[rgb]{0.56,0.35,0.01}{\textbf{\textit{#1}}}}
\usepackage{graphicx}
\makeatletter
\def\maxwidth{\ifdim\Gin@nat@width>\linewidth\linewidth\else\Gin@nat@width\fi}
\def\maxheight{\ifdim\Gin@nat@height>\textheight\textheight\else\Gin@nat@height\fi}
\makeatother
% Scale images if necessary, so that they will not overflow the page
% margins by default, and it is still possible to overwrite the defaults
% using explicit options in \includegraphics[width, height, ...]{}
\setkeys{Gin}{width=\maxwidth,height=\maxheight,keepaspectratio}
% Set default figure placement to htbp
\makeatletter
\def\fps@figure{htbp}
\makeatother
\setlength{\emergencystretch}{3em} % prevent overfull lines
\providecommand{\tightlist}{%
  \setlength{\itemsep}{0pt}\setlength{\parskip}{0pt}}
\setcounter{secnumdepth}{-\maxdimen} % remove section numbering
\ifLuaTeX
  \usepackage{selnolig}  % disable illegal ligatures
\fi
\IfFileExists{bookmark.sty}{\usepackage{bookmark}}{\usepackage{hyperref}}
\IfFileExists{xurl.sty}{\usepackage{xurl}}{} % add URL line breaks if available
\urlstyle{same}
\hypersetup{
  pdftitle={Assignment 2: Coding Basics},
  pdfauthor={Ayoung Kim},
  hidelinks,
  pdfcreator={LaTeX via pandoc}}

\title{Assignment 2: Coding Basics}
\author{Ayoung Kim}
\date{}

\begin{document}
\maketitle

\hypertarget{overview}{%
\subsection{OVERVIEW}\label{overview}}

This exercise accompanies the lessons in Environmental Data Analytics on
coding basics.

\hypertarget{directions}{%
\subsection{Directions}\label{directions}}

\begin{enumerate}
\def\labelenumi{\arabic{enumi}.}
\tightlist
\item
  Rename this file
  \texttt{\textless{}FirstLast\textgreater{}\_A02\_CodingBasics.Rmd}
  (replacing \texttt{\textless{}FirstLast\textgreater{}} with your first
  and last name).
\item
  Change ``Student Name'' on line 3 (above) with your name.
\item
  Work through the steps, \textbf{creating code and output} that fulfill
  each instruction.
\item
  Be sure to \textbf{answer the questions} in this assignment document.
\item
  When you have completed the assignment, \textbf{Knit} the text and
  code into a single PDF file.
\item
  After Knitting, submit the completed exercise (PDF file) to Sakai.
\end{enumerate}

\hypertarget{basics-part-1}{%
\subsection{Basics, Part 1}\label{basics-part-1}}

\begin{enumerate}
\def\labelenumi{\arabic{enumi}.}
\item
  Generate a sequence of numbers from one to 30, increasing by threes.
  Assign this sequence a name.
\item
  Compute the mean and median of this sequence.
\item
  Ask R to determine whether the mean is greater than the median.
\item
  Insert comments in your code to describe what you are doing.
\end{enumerate}

\begin{Shaded}
\begin{Highlighting}[]
\CommentTok{\#1. Sequence 1 (I defined seq(1,30.3) as sequence 1)}
\FunctionTok{seq}\NormalTok{(}\DecValTok{1}\NormalTok{,}\DecValTok{30}\NormalTok{,}\DecValTok{3}\NormalTok{)}
\end{Highlighting}
\end{Shaded}

\begin{verbatim}
##  [1]  1  4  7 10 13 16 19 22 25 28
\end{verbatim}

\begin{Shaded}
\begin{Highlighting}[]
\NormalTok{sequence1}\OtherTok{\textless{}{-}}\FunctionTok{seq}\NormalTok{(}\DecValTok{1}\NormalTok{,}\DecValTok{30}\NormalTok{,}\DecValTok{3}\NormalTok{)}


\CommentTok{\#2. Mean and Median (I got mean and median of Sequence 1 that I defined above)}

\FunctionTok{mean}\NormalTok{(sequence1)}
\end{Highlighting}
\end{Shaded}

\begin{verbatim}
## [1] 14.5
\end{verbatim}

\begin{Shaded}
\begin{Highlighting}[]
\FunctionTok{median}\NormalTok{(sequence1)}
\end{Highlighting}
\end{Shaded}

\begin{verbatim}
## [1] 14.5
\end{verbatim}

\begin{Shaded}
\begin{Highlighting}[]
\CommentTok{\#3. Mean \textgreater{} Median ? (I got 14.5 for both mean and median)}
\FloatTok{14.5}\SpecialCharTok{\textgreater{}}\FloatTok{14.5}
\end{Highlighting}
\end{Shaded}

\begin{verbatim}
## [1] FALSE
\end{verbatim}

\hypertarget{basics-part-2}{%
\subsection{Basics, Part 2}\label{basics-part-2}}

\begin{enumerate}
\def\labelenumi{\arabic{enumi}.}
\setcounter{enumi}{4}
\item
  Create a series of vectors, each with four components, consisting of
  (a) names of students, (b) test scores out of a total 100 points, and
  (c) whether or not they have passed the test (TRUE or FALSE) with a
  passing grade of 50.
\item
  Label each vector with a comment on what type of vector it is.
\item
  Combine each of the vectors into a data frame. Assign the data frame
  an informative name.
\item
  Label the columns of your data frame with informative titles.
\end{enumerate}

\begin{Shaded}
\begin{Highlighting}[]
\CommentTok{\#test score}

\CommentTok{\#Name of Student}
\NormalTok{student }\OtherTok{=} \FunctionTok{c}\NormalTok{(}\StringTok{"Anne"}\NormalTok{,}\StringTok{"Marie"}\NormalTok{,}\StringTok{"Jack"}\NormalTok{,}\StringTok{"Wilson"}\NormalTok{)}

\CommentTok{\#Test score}
\NormalTok{testscore }\OtherTok{=} \FunctionTok{c}\NormalTok{(}\DecValTok{45}\NormalTok{,}\DecValTok{70}\NormalTok{,}\DecValTok{80}\NormalTok{,}\DecValTok{90}\NormalTok{)}

\CommentTok{\#Pass or Fail}
\NormalTok{pass}\OtherTok{\textless{}{-}}\NormalTok{(testscore}\SpecialCharTok{\textgreater{}}\DecValTok{50}\NormalTok{)}

\CommentTok{\#Test score/student name/pass}
\NormalTok{test\_score}\OtherTok{\textless{}{-}}\NormalTok{testscore}
\NormalTok{student\_name }\OtherTok{\textless{}{-}}\NormalTok{student}
\NormalTok{pass}\OtherTok{\textless{}{-}}\NormalTok{pass}

\CommentTok{\#Data frame for Test Score}
\FunctionTok{class}\NormalTok{(test\_score)}
\end{Highlighting}
\end{Shaded}

\begin{verbatim}
## [1] "numeric"
\end{verbatim}

\begin{Shaded}
\begin{Highlighting}[]
\NormalTok{df\_test\_score }\OtherTok{\textless{}{-}} \FunctionTok{as.data.frame}\NormalTok{(test\_score)}
\NormalTok{df\_test\_score}
\end{Highlighting}
\end{Shaded}

\begin{verbatim}
##   test_score
## 1         45
## 2         70
## 3         80
## 4         90
\end{verbatim}

\begin{Shaded}
\begin{Highlighting}[]
\CommentTok{\#Data frame for Student Names}
\FunctionTok{class}\NormalTok{(student\_name)}
\end{Highlighting}
\end{Shaded}

\begin{verbatim}
## [1] "character"
\end{verbatim}

\begin{Shaded}
\begin{Highlighting}[]
\NormalTok{df\_student\_name }\OtherTok{\textless{}{-}}\FunctionTok{as.data.frame}\NormalTok{(student\_name)}
\NormalTok{df\_student\_name}
\end{Highlighting}
\end{Shaded}

\begin{verbatim}
##   student_name
## 1         Anne
## 2        Marie
## 3         Jack
## 4       Wilson
\end{verbatim}

\begin{Shaded}
\begin{Highlighting}[]
\CommentTok{\#Data frame for \textquotesingle{}Pass or Fail\textquotesingle{}}
\FunctionTok{class}\NormalTok{(pass)}
\end{Highlighting}
\end{Shaded}

\begin{verbatim}
## [1] "logical"
\end{verbatim}

\begin{Shaded}
\begin{Highlighting}[]
\NormalTok{df\_pass }\OtherTok{\textless{}{-}}\FunctionTok{as.data.frame}\NormalTok{(pass)}
\NormalTok{df\_pass}
\end{Highlighting}
\end{Shaded}

\begin{verbatim}
##    pass
## 1 FALSE
## 2  TRUE
## 3  TRUE
## 4  TRUE
\end{verbatim}

\begin{Shaded}
\begin{Highlighting}[]
\CommentTok{\#Adding Columns }
\NormalTok{df}\OtherTok{\textless{}{-}}\FunctionTok{cbind}\NormalTok{(df\_student\_name,df\_test\_score,df\_pass)}
\FunctionTok{class}\NormalTok{(df)}
\end{Highlighting}
\end{Shaded}

\begin{verbatim}
## [1] "data.frame"
\end{verbatim}

\begin{Shaded}
\begin{Highlighting}[]
\NormalTok{df}
\end{Highlighting}
\end{Shaded}

\begin{verbatim}
##   student_name test_score  pass
## 1         Anne         45 FALSE
## 2        Marie         70  TRUE
## 3         Jack         80  TRUE
## 4       Wilson         90  TRUE
\end{verbatim}

\begin{Shaded}
\begin{Highlighting}[]
\FunctionTok{names}\NormalTok{(df)}\OtherTok{\textless{}{-}}\StringTok{"Test Scores of Students"}

\CommentTok{\#Name}
\FunctionTok{names}\NormalTok{(df\_student\_name)}\OtherTok{\textless{}{-}}\StringTok{"Student"}
\FunctionTok{names}\NormalTok{(df\_test\_score)}\OtherTok{\textless{}{-}}\StringTok{"Score"}
\FunctionTok{names}\NormalTok{(df\_pass)}\OtherTok{\textless{}{-}}\StringTok{"Pass"}
\end{Highlighting}
\end{Shaded}

\begin{enumerate}
\def\labelenumi{\arabic{enumi}.}
\setcounter{enumi}{8}
\tightlist
\item
  QUESTION: How is this data frame different from a matrix?
\end{enumerate}

\begin{quote}
Answer:\#Answer: with a data frame, I can combine data sets and show in
a one chart.
\end{quote}

\begin{enumerate}
\def\labelenumi{\arabic{enumi}.}
\setcounter{enumi}{9}
\tightlist
\item
  Create a function with an if/else statement. Your function should take
  a \textbf{vector} of test scores and print (not return) whether a
  given test score is a passing grade of 50 or above (TRUE or FALSE).
  You will need to choose either the \texttt{if} and \texttt{else}
  statements or the \texttt{ifelse} statement.
\end{enumerate}

\#Function for Marie, Jack, and Wilson df\_test\_score\textless-x
if(x\textgreater50)\{ ``Pass'' \} \#Function for Anne
df\_test\_score\textless-x if(x\textless50)\{ `fail' \} 11. Apply your
function to the vector with test scores that you created in number 5.

\begin{Shaded}
\begin{Highlighting}[]
\CommentTok{\#Test score for Anne}
\NormalTok{df\_test\_score }\OtherTok{\textless{}{-}}\DecValTok{45}
\ControlFlowTok{if}\NormalTok{ (df\_test\_score}\SpecialCharTok{\textless{}}\DecValTok{50}\NormalTok{)\{}
  \StringTok{\textquotesingle{}Fail\textquotesingle{}}
\NormalTok{\}}
\end{Highlighting}
\end{Shaded}

\begin{verbatim}
## [1] "Fail"
\end{verbatim}

\begin{Shaded}
\begin{Highlighting}[]
\CommentTok{\#Test score for Marie}
\NormalTok{df\_test\_score}\OtherTok{\textless{}{-}}\DecValTok{70}
\ControlFlowTok{if}\NormalTok{(df\_test\_score}\SpecialCharTok{\textgreater{}}\DecValTok{50}\NormalTok{)\{}
  \StringTok{\textquotesingle{}pass\textquotesingle{}}
\NormalTok{\}}
\end{Highlighting}
\end{Shaded}

\begin{verbatim}
## [1] "pass"
\end{verbatim}

\begin{Shaded}
\begin{Highlighting}[]
\CommentTok{\#Test score for Jack}
\NormalTok{df\_test\_score }\OtherTok{\textless{}{-}}\DecValTok{80}
\ControlFlowTok{if}\NormalTok{(df\_test\_score}\SpecialCharTok{\textgreater{}}\DecValTok{50}\NormalTok{)\{}
  \StringTok{\textquotesingle{}pass\textquotesingle{}}
\NormalTok{\}}
\end{Highlighting}
\end{Shaded}

\begin{verbatim}
## [1] "pass"
\end{verbatim}

\begin{Shaded}
\begin{Highlighting}[]
\CommentTok{\#Test score for Wilson}
\NormalTok{df\_test\_score}\OtherTok{\textless{}{-}}\DecValTok{90}
\ControlFlowTok{if}\NormalTok{(df\_test\_score}\SpecialCharTok{\textgreater{}}\DecValTok{50}\NormalTok{)\{}
  \StringTok{\textquotesingle{}pass\textquotesingle{}}
\NormalTok{\}}
\end{Highlighting}
\end{Shaded}

\begin{verbatim}
## [1] "pass"
\end{verbatim}

\begin{enumerate}
\def\labelenumi{\arabic{enumi}.}
\setcounter{enumi}{11}
\tightlist
\item
  QUESTION: Which option of \texttt{if} and \texttt{else}
  vs.~\texttt{ifelse} worked? Why?
\end{enumerate}

\begin{quote}
Answer:The `if' option worked for me. I used different function for
Anne's score.I put x\textless50 for Anne's score and x\textgreater50 for
the scores of Marie, Jack, and Wilson.I thought, for Anne's score, the
combination of `if' and `else' would work, but it didn't. so I decided
to put x\textless50 for Anne's case.
\end{quote}

\end{document}
